\documentclass[doc]{apa7}
\usepackage{amsmath}
\usepackage{graphicx}
\newcommand\scalemath[2]{\scalebox{#1}{\mbox{\ensuremath{\displaystyle #2}}}}
\title{EM Algorithm Details}
\author{Xiang Liu}
\affiliation{Educational Testing Service}


\begin{document}
\maketitle
	\section{Derivative identities}
	We first establish some identities for derivatives that will be used
  repeatedly.
  The probability of observing a tool use is given by the following response
  function,
  \begin{equation}
    P(W_{ij} = 1 | \theta_{i2}) = \frac{1}{1 + \exp(-\alpha_j
    (\theta_{i2} - \beta_j))} = P_{ij}.
  \end{equation}
  The probability of not observing a tool use is then 
  \begin{equation}
    P(W_{ij} = 0 | \theta_{i2}) = 1 - P_{ij} = Q_{ij}.
  \end{equation}
  The first-order partial derivatives are 
  \begin{equation}
    \frac{\partial P_{ij}}{\partial \alpha_j} = \frac{\theta_{i2}\exp(-\alpha_j
    (\theta_{i2}-\beta_j))}{(1+\exp(-\alpha_j(\theta_{i2}-\beta_j)))^2} =
    (\theta_{i2}-\beta_j) P_{ij} Q_{ij},
  \end{equation}
  and
  \begin{equation}
    \frac{\partial P_{ij}}{\partial \beta_j} = -\frac{\alpha_j\exp(-\alpha_j
    (\theta_{i2}-\beta_j))}{(1+\exp(-\alpha_j(\theta_{i2}-\beta_j)))^2} =
    -\alpha_j P_{ij} Q_{ij}.
  \end{equation}
  The probability of getting a correct response conditional on $W_{ij} = w_{ij}$
  is
  \begin{equation}
    P(X_{ij} = 1 | \theta_{i1}, w_{ij}) = \frac{1}{1 + \exp(-a_j (\theta_{i1} -
    b_j + w_{ij} \gamma_j))} = \dot{P_{ij}}.
  \end{equation}
  Then the probability of getting an incorrect response is
  \begin{equation}
    P(X_{ij} = 0 | \theta_{i1}, w_{ij}) = 1 - \dot{P_{ij}} = \dot{Q_{ij}}.
  \end{equation}
  Taking first-order partial derivatives, it leads to
  \begin{equation}
    \frac{\partial \dot{P_{ij}}}{\partial a_j} = (\theta_{i1}-b_j+w_{ij}\gamma_j)
    \dot{P_{ij}}\dot{Q_{ij}},
  \end{equation}
  \begin{equation}
    \frac{\partial \dot{P_{ij}}}{\partial b_j} = -a_j\dot{P_{ij}}\dot{Q_{ij}},
  \end{equation}
  and
  \begin{equation}
    \frac{\partial \dot{P_{ij}}}{\partial \gamma_j} = a_jw_{ij}\dot{P_{ij}}
    {\dot{Q_{ij}}}.
  \end{equation}

  \section{Likelihood Equations}
  Given $\bm{X} = \bm{x}$, $\bm{W} = \bm{w}$, and $\bm{\Theta} = \bm{\theta}$
  the complete data likelihood of the model parameters are
  \begin{equation}
    L = \prod_{i=1}^{N} \phi(\bm{\theta}_i;\rho)\prod_{j=1}^{J} P_{ij}^{w_{ij}} Q_{ij}^{1-w_{ij}}
    \dot{P_{ij}}^{x_{ij}} \dot{Q_{ij}}^{1-x_{ij}}.
  \end{equation}
  Equivalently, the compete data log-likelihood is
  \begin{equation}
    \log L = \sum_{j=1}^{J}l_j(\alpha_j, \beta_j) + \sum_{j=1}^{J}l_j(a_j,
    b_j,\gamma_j) + \sum_{i=1}^{N} \log \phi(\bm{\theta}_i;\rho),
  \end{equation}
  where
  \begin{equation}
    l_j(\alpha_j, \beta_j) = \sum_{i=1}^{N} w_{ij}\log P_{ij} + (1-w_{ij})\log
    Q_{ij},
  \end{equation}
  and
  \begin{equation}
    l_j(a_j,b_j,\gamma_j) = \sum_{i=1}^{N} x_{ij}\log \dot{P_{ij}} + (1-x_
    {ij})\log \dot{Q_{ij}}.
  \end{equation}
  Since the three sets of item parameters, $\{\alpha_j, \beta_j\}$, $\{a_j,
  b_j, \gamma_j\}$, and $\{\rho\}$, are separated in different sums in the
  complete data log-likelihood, we take gradient separately, i.e.
  \begin{equation}
  \label{eq:lik_eq_alpha_beta}
    \nabla l_j(\alpha_j, \beta_j) = 
    \begin{pmatrix}
      \frac{\partial l_j}{\partial \alpha_j}\\
      \\
      \frac{\partial l_j}{\partial \beta_j}
    \end{pmatrix}=
    \begin{pmatrix}
      \sum_{i=1}^{N} (w_{ij}-P_{ij})(\theta_{i2}-\beta_j)\\
      \\
      -\sum_{i=1}^{N} (w_{ij}-P_{ij})\alpha_j
    \end{pmatrix},
  \end{equation}
  and
  \begin{equation}
  \label{eq:lik_eq_a_b_gamma}
    \nabla l_j(a_j,b_j,\gamma_j) =
    \begin{pmatrix}
      \frac{\partial l_j}{\partial a_j}\\
      \\
      \frac{\partial l_j}{\partial b_j}\\
      \\
      \frac{\partial l_j}{\partial \gamma_j}
    \end{pmatrix}=
    \begin{pmatrix}
      \sum_{i=1}^{N} (x_{ij} - P_{ij})(\theta_{i1}-b_j+w_{ij}\gamma_j)\\
      \\
      -\sum_{i=1}^{N} (x_{ij} - P_{ij})a_j\\
      \\
      \sum_{i=1}^{N} (x_{ij} - P_{ij})a_j w_{ij}
    \end{pmatrix}.
  \end{equation}
  Taking the first order partial derivative w.r.t $\rho$ leads to a third order
  polynomial in $\rho$,
  \begin{equation}
  \label{eq:lik_eq_rho}
    \frac{\partial l(\rho)}{\partial \rho} = N\rho^3-\left(\sum_{i=1}^
    {N}\theta_{i1}\theta_{i2}\right)\rho^2 - \left(N - \sum_{i=1}^{N}
    (\theta_{i1}^2+\theta_{i2}^2)\right)\rho - \sum_{i=1}^{N}\theta_{i1}\theta_{i2}
  \end{equation}

  \section{Jacobian Matrix}
  In finding the maximum likelihood estimates, we need to solve the
  likelihood equations,
  \begin{equation}
    E_{\bm{\theta}|\dots}\left[\nabla l_j(\alpha_j, \beta_j)\right] = \bm{0},
  \end{equation}
  \begin{equation}
    E_{\bm{\theta}|\dots}\left[\nabla l_j(a_j, b_j, \gamma_j)\right] = \bm{0},
  \end{equation},
  and
  \begin{equation}
    E_{\bm{\theta}|\dots}\left[\frac{\partial l(\rho)}{\partial \rho}\right] = 0.
  \end{equation}
  These nonlinear systems of equations can be solved numerically by the
  derivative based Newton-Raphson method. The Jacobian matrix of the
  gradients are
  \begin{equation}
  \begin{gathered}
    \bm{J}_{\nabla l_j(\alpha_j, \beta_j)} = 
    \begin{pmatrix}
      \frac{\partial^2 l_j}{\partial \alpha_j^2} &  \frac{\partial^2 l_j}
      {\partial \alpha_j \partial \beta_j}\\
      \\
      \frac{\partial^2 l_j}{\partial \beta_j \partial \alpha_j} &  
      \frac{\partial^2 l_j}{\partial \beta_j^2}\\
    \end{pmatrix}=\\
    \begin{pmatrix}
      -\sum_i(\theta_{i2}-\beta_j)^2P_{ij}Q_{ij} & -\sum_i
      (w_{ij}-P_{ij})-(\theta_{i2}-\beta_j)\alpha_jP_{ij}Q_{ij}\\
      \\
      -\sum_i(w_{ij}-P_{ij}) - (\theta_{i2}-\beta_j)\alpha_jP_{ij}Q_{ij}
      & -\sum_i\alpha_j^2 P_{ij} Q_{ij}
    \end{pmatrix}
  \end{gathered}
  \end{equation}
  and
  \begin{equation}
    \begin{gathered}
      \bm{J}_{\nabla l_j(a_j, b_j, \gamma_j)} = 
      \begin{pmatrix}
        \frac{\partial^2 l_j}{\partial a_j^2} & 
        \frac{\partial^2 l_j}{\partial a_j \partial b_j} &
        \frac{\partial^2 l_j}{\partial a_j \partial \gamma_j}\\
        \frac{\partial^2 l_j}{\partial b_j \partial a_j} &
        \frac{\partial^2 l_j}{\partial b_j^2} &
        \frac{\partial^2 l_j}{\partial b_j \partial \gamma_j}\\
        \frac{\partial^2 l_j}{\partial \gamma_j \partial a_j} &
        \frac{\partial^2 l_j}{\partial \gamma_j \partial b_j} &
        \frac{\partial^2 l_j}{\partial \gamma_j^2}
      \end{pmatrix} = \\
      \scalemath{0.55}{
      \begin{pmatrix}
        -\sum_i(\theta_{i1}-b_j+w_{ij}\gamma_j)^2 \dot{P_{ij}}\dot{Q_{ij}} &
        \sum_i(\theta_{i1}-b_j+w_{ij}\gamma_j)a_j\dot{P_{ij}}\dot{Q_{ij}} - 
        (x_{ij}-\dot{P_{ij}}) &
        -\sum_i(\theta_{i1}-b_j+w_{ij}\gamma_j)a_jw_{ij}\dot{P_{ij}}\dot{Q_
        {ij}}-w_{ij}(x_{ij}-\dot{P_{ij}})\\
        \\
        \sum_i(\theta_{i1}-b_j+w_{ij}\gamma_j)a_j\dot{P_{ij}}\dot{Q_{ij}}-(x_
        {ij}-\dot{P_{ij}}) &
        -\sum_i a_j^2 \dot{P_{ij}}\dot{Q_{ij}} &
        \sum_i a_j^2 w_{ij} \dot{P_{ij}} \dot{Q_{ij}}\\
        \\
        -\sum_i (\theta_{i1}-b_j+w_{ij}\gamma_j)a_jw_{ij}\dot{P_{ij}}\dot{Q_
        {ij}} - w_{ij}(x_{ij}-\dot{P_{ij}}) &
        \sum_i a_j^2 w_{ij} \dot{P_{ij}} \dot{Q_{ij}} &
        -\sum_i a_j^2 w_{ij}^2 \dot{P_{ij}} \dot{Q_{ij}} 
      \end{pmatrix}
      }
    \end{gathered}
  \end{equation}






\end{document}